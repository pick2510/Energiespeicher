%%This is a very basic article template.
%%There is just one section and two subsections.
\RequirePackage[ngerman=ngerman-x-latest]{hyphsubst}
%\documentclass[11pt,a4paper]{scrbook}
\documentclass[11pt,a4paper]{scrartcl}
%\documentclass[11pt,a4paper]{article}
\usepackage[utf8]{inputenc}
\usepackage[T1]{fontenc}
%\renewcommand{\familydefault}{\sfdefault}
\usepackage[a4paper]{geometry}
%\geometry{verbose,tmargin=2.5cm,lmargin=2.5cm,rmargin=1.5cm,bmargin=3cm}
\usepackage[ngerman,english]{babel}
%\usepackage{ngerman}
\usepackage{amsmath}
\usepackage{amsfonts}
\usepackage{amssymb}
\usepackage{setspace}
\usepackage[pdftex]{graphicx}
\usepackage{epstopdf}
\usepackage[final]{pdfpages}
\usepackage{chngcntr}
%\usepackage{hyperref}
\usepackage{placeins}

\setlength{\parindent}{0pt}
\setlength{\headheight}{14pt}
\usepackage{fancyhdr}
\pagestyle{fancy}
\begin{document}

\begin{titlepage}
\begin{center}
\includegraphics[scale=0.8]{images/HSR.pdf}
\linebreak \includegraphics[scale=0.3]{images/IET.pdf}
\end{center}

\vspace{2.5cm}
%\vspace{0.5cm}
\begin{center}
\textbf{Latentwärmespeicher und chemische Speicher
in Gebäudeenergieversorgungssystemen}
\linebreak
%\textbf{evtl. Vertraulich}
\end{center}
\vspace{1.8cm}
%\vspace{0.5cm}
\begin{center}
Seminararbeit
\end{center}
\vspace{1cm}
\begin{center}
\textbf{von \linebreak Dominik Strebel \linebreak Simon Boller \linebreak
Leandro Nikolic} \linebreak
\linebreak 
Abgabedatum: 09.05.2014
\linebreak
\end{center}
\vspace{3cm}
\noindent Betreuung:

\noindent Prof. Carsten Wemhöner

\noindent HSR Rapperswil

\noindent Institut für Energietechnik

\end{titlepage}
%\thispagestyle{empty}
%\cleardoublepage
\renewcommand{\footrulewidth}{0pt}
\renewcommand{\headrulewidth}{0pt}
\lhead{}
\chead{}
\rhead{}
\cfoot{} 
\selectlanguage{ngerman}


 \vspace*{12.5cm}
\begin{minipage}{80mm}
	Keywords: Wärmespeicher, Latentwärmespeicher, chemische Speicher, Adsorption,
 \\
	\\
	Zitiervorschlag: 
	Strebel et al. Latentwärmespeicher und chemische Speicher in
	Gebäudeenergieversorgungssystemen. Seminararbeit. unveröffentlicht. 2014
	\vspace{1cm}


  \rule{80mm}{2pt}
  Impressum: \\
  Hochschule für Technik Rapperswil \\
  IET, Institut für Energietechnik \\ 
  Oberseestrasse 10 \\
  8640 Rapperswil\\
  \rule{80mm}{2pt}
\end{minipage}
\newpage


\tableofcontents
\newpage
\renewcommand{\headrulewidth}{0.4pt}
\renewcommand{\footrulewidth}{0.4pt}
\lhead{}
\chead{Seminararbeit}
\rhead{}
\cfoot{}
\setcounter{page}{1}
\cfoot{\thepage}
\section{Einleitung}
\newpage
\section{Allgemeiner Vergleich von chemischen, latenten und sensiblen
Wärmespeichern}
Wärmespeicher lassen sich generell in zwei verschiedene Hauptgruppen einteilen.
Einerseites existieren chemische Energiespeicher, andererseits
direkt-thermische, in denen die Energie ohne Umwandlung als thermische Energie
verfügbar ist. Eine Gliederung der verschiedenen Technologien befindet sich in
der Abbildung
\ref{fig:Wärmespeicher}

\begin{figure}[h]
\begin{center}
\includegraphics[scale=0.3]{images/speicher.jpg}
\caption{Übersicht über die verschiedenen Wärmespeichertechnolgien \cite{BINE}}
\label{fig:Wärmespeicher}
\end{center}
\end{figure}

\subsection{Chemische Speicher}
Chemische Speicher werden über eine chemsiche Reaktion be- und entladen. Per
Definition eines Speichers sind diese Reaktionen reversibel. Die nutzbare
Wärmeenergie entspricht der freigesetzten Reaktionsenthalpie $\Delta
H_{\mathrm{R}}$. Die Nutztemperatur des Speichers bestimmt die in Frage
kommenden chemischen Reaktionen und schränkt die zu verwenden Stoffe erheblich
ein.

Chemische Speicher werden heute im allgemeinen mit sorptiven Prozessen gebaut.
Dies ist einerseits die Adsorption, eine Anlagerung eines Gases oder einer
Flüssigkeit an einen Feststoff und andererseits die Absorption, das Lösen von
Gasen in einer Flüssigkeit. In untenstehender Gleichung ist das Grundlegende
Prinzip der Adsorption erklärt.
\begin{align}
\text{Sorbens}+nH_2O\leftrightharpoons \text{Sorbens}+nH_2O+\Delta H_{ads}
\end{align}
Nutzbar ist die freiwerdende Reaktionsenthalpie $\Delta H_{ads}$ wenn die
Reaktion nach links erfolgt. Die Beladung erfolgt nach umgekehrten Prinzip.
In Abbildung \ref{fig:Sorption} ist ein typischer wasserbasierter
Adsorptionsspeicher dargestellt. Zum Beladen des Speichers wird der desorbierte
Wasserdampf im Kondensator kondensiert. Entladen wird der Speicher mit
niedrigtemperatur Verdampfung von Wasser, der Dampf wird anschliessend an ein
Sorptionsmedium adsorbiert. Dabei wird Hochtemperaturwärme freigesetzt.
\cite{Wesselak}

\begin{figure}[h]
\begin{center}
\includegraphics[scale=1]{images/sorption.pdf}
\caption{Prinzip eines geschlossenen wasserbasierten Soroptionsprinzip
\cite{Wesselak}}
\label{fig:Sorption}
\end{center}
\subsection{Latente Speicher}
Latentwärmespeicher nutzen den Phasenübergang eines Stoffs zur Speicherung von
Wärme. Wie bei den chemischen Speichern beschränken sich die Einsatzgebiete 
\end{figure}
\section{Technologischer Überblick}
\subsection{Latentspeicher}
\subsubsection{Stand der Technik}
\subsubsection{Entwicklung}
\subsubsection{Perspektiven}
\subsubsection{Chancen}
\subsubsection{Herausforderungen}
\subsection{chemische Speicher}
\subsubsection{Stand der Technik}
\subsubsection{Entwicklung}
\subsubsection{Perspektiven}
\subsubsection{Chancen}
\subsubsection{Herausforderungen}


\newpage
\section{Einsatzgebiete}
\subsection{Latentwärmespeicher}
\subsection{chemische Speicher}

\newpage
\section{Ausblick}

\listoftables
\newpage
\listoffigures
\newpage
\begin{thebibliography}{99}
	\bibitem{BINE}BINE Energieforschung für die Praxis. Abgerufen am 11.04.2014 von
	http://www.bine.info/typo3temp/pics/b193b0972c.gif verändert durch D. Strebel
	\bibitem{Wesselak}Wesselak et. al. Regenerative Energiesysteme. 2. Auflage.
	Springer Vieweg Verlag Berlin. 2013
\end{thebibliography}
\end{document}
